\section{Research Contributions}
\label{sec:research_contributions}

%\subsection{Gaps}
%\label{subsub:gaps}
%The previous section addressed the research projects related to the control and management of resilience that leverages the power of SDN to propose centralized solutions, including those that apply to data mining. Below are presented the most relevant gaps encountered in related work.

%In this vein, recent research has proposed TE mechanisms that face particular challenges of SDN resilience as discussed in section xxx. From where we have that the techniques of Analysis/Characterization of traffic are closely related to the other approaches TE and with strategies of resilience such as detection of failures and prediction of the congestion of the links \cite{ian_2014:a_road_map_sdn}. In this way, the techniques of Analysis/Characterization become a crucial component in the management of the resilience. However, solutions that analyze the behavior of SDN traffic introduce significant overhead into the network \cite{Yu_2013:flow_sense}. Therefore, proposals such as \cite{chowdhury_2014:payless, Yu_2013:flow_sense, minlan_2013:OpenSketch, Tootoonchian_2010:opentm} seek more efficient Analysis / Characterization mechanisms to achieve high accuracy and low network overhead. Despite the significant advances shown by these proposals, Traffic Analysis/Characterization in SDN continues to be a major research challenge. The challenge that increases if the Analysis/Characterization is done at or near the execution time.

\subsection{Research Contributions}
\label{subsub:research_contributions}
The present thesis proposal aims to achieve the following contributions:

%\begin{itemize}
    %\item A high-level framework that integrates the concepts of ML and SDN into the design of a DCN topology---potentially, switch-centric---for enabling the development of an effective load-balancing solution.
    %\item A mechanism that uses one or a combination of ML techniques for fine-granularity prediction of flow characteristics---potentially, size and time length---in a DCN with the appropriate accuracy and performance requirements.
    %\item A multipath routing mechanism based on SDN that uses the flow characteristics predicted by the ML mechanism for minimizing the maximum load of links---optimizing load-balancing---in a DCN aiming to provide high throughput and low delay while maintaining efficient use of resources.
%\end{itemize}