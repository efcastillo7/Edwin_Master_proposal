\section{Problem Statement}
\label{sec:problem_statement}

%The exponential growth of IP traffic makes traditional networks very difficult to manage \cite{Jararweh_2015:sdn_iot}. This difficulty lies in the fact that administrators must deal with complex communication protocols (vendor-specific), network policy diffuse deployment and limited routing scalability \cite{Sterbenz_2013:taxonomy_n, anderson_da_silva_2015:resilience_sdn_survey, Klein_2013:of_omnet}. Even more, the network devices are vertically integrated: control plane and data plane embedded in the same hardware. Software-Defined-Networking (SDN) is a paradigm that promises to solve such difficulties, redefining the network architecture by separating between the control plane and the data plane, centralizing control logic and introducing the ability to program network \cite{kreutz_2015:sdn_comprehensive_survey, nunes_2014:surve_sdn_ppf}.

SDN provides a flexible architecture fast and easy configuration of network devices. In addition, it implements fine-grained network management based on the decoupling of the data and control planes \cite{herrera_2016:nfv_survey}. However, this separation is not enough to guarantee that the network will not degrade with the increment of traffic. In this sense, another important approach to optimize a network and improve network robustness is Traffic Engineering (TE) \cite{ian_2014:a_road_map_sdn}.

TE deals with measurement and management of network traffic aiming to improve the utilization of network resources and the associated Quality of Service (QoS). TE, in SDN, focuses on four approaches \cite{ian_2014:a_road_map_sdn, z_shu_2016:TE_SDN_measure_manage}: (\textit{i}) Flow Management that seeks to find both the solution to prevent network overload and the trade-off between latency and load balancing, (\textit{ii}) Fault Tolerance intends to ensure the immediate recovery of the network when a failure occurs in any node, (\textit{iii}) Topology Update aims to update the network policies in real time and ensure their application in each flow; and (\textit{iv}) Analysis/Characterization of traffic that is related to mechanisms for network monitoring, debugging, fault detection, and data collection.

An essential requirement for TE is to provide accurate and reliable traffic monitoring (e.g., Flow Management needs granular real-time monitoring information to compute the most efficient routing decisions). Traffic monitoring is fundamental to observe and quantify what is happening in the network, and it is closely related to Traffic Management mechanisms \cite{tangari_2017:decentralized_monitoring, machado_2014:towards_SLA_Policy}. For example, policy rules can be triggered by conditions monitored in the network (e.g., link congestion or bottleneck). Network traffic monitoring techniques can be classified into two categories \cite{mohan2011:active_passive,Ningning_2003:probing_techniques}: passive monitoring and active probing (also called active monitoring). Passive monitoring tools use the trace history of existing data transmission to analyze network health ``capture-and-analyze". Active probing, on the other hand, injects test packages to the network, measures the response and checks the result for important factors (e.g., latency, jitter, throughput, packet loss).

In the literature, few proposals \cite{suh_2014:OpenSample,Yu_2013:flow_sense} use passive tools to network traffic monitoring in SDN. While the results efficient, accurate and provide fine-grained analysis, their scope is limited to expensive instrumentation and infrastructure. Approaches like \cite{bandi_2007:JFlow, Sflow_2003:sflow, Cisco_2012:netflow}, on the other hand, use traditional Probing-based tools (used in IP networks) for traffic monitoring in SDN. These approaches have some shortcomings such as the increased overhead incurred by statistics collection from the whole network in large-scale SDN-based networks (e.g., Data Centers). Besides, they require specialized and proprietary instrumentation on network devices. These limitations are due because traditional Probing-based tools were not designed to cope with the context of SDN.

To overcome, the shortcomings above mentioned some approach \cite{chowdhury_2014:payless,raumer_2014:monsamp, van_2014:OpenNetMon,Tootoonchian_2010:opentm,Sun_2015:HONE,Dusi_2014:reactive_sdn, jose_2011:online, minlan_2013:OpenSketch} introduce more efficient Probing tools. In \cite{chowdhury_2014:payless,raumer_2014:monsamp,van_2014:OpenNetMon,Tootoonchian_2010:opentm} the authors proposed techniques that increment the network overhead while optimizing measurement accuracy (accuracy increases at the expense of overhead). In \cite{Sun_2015:HONE, Dusi_2014:reactive_sdn,jose_2011:online} increase accuracy and minimize network overhead by the insertion of additional modules in the network devices (increases accuracy at the expense of adding resources). In \cite{minlan_2013:OpenSketch} the authors design a new Probing-based protocol, parallel to OpenFlow, to achieve monitoring in SDNs. A new SDN protocol, however, requires an upgrade or replacement of all network nodes, a significant investment ISPs will be reluctant to make. Furthermore, standardization of a new protocol has shown to be a long and tedious task. 

Notwithstanding, the significant contributions of the approaches aforementioned, they also present critical shortcomings as imbalance both in the overhead-resource/accuracy on Probing-based Monitoring. Besides, they lack intelligent mechanisms that can predict the value of the measured parameters to optimize probing frequency. Which therefore minimizes traffic overhead while providing high prediction accuracy. As a result of these shortcomings, intelligent Probing for SDN Monitoring is a major research challenge. Therefore, this proposal focuses on solving the following research question:


\begin{center}
\textbf{How to probing SDN in an intelligent way with a high accuracy and with a negligible overhead?}
\end{center}